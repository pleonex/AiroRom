% Copyright (c) 2015 Benito Palacios S�nchez - All Rights Reserved.
% Esta obra est� licenciada bajo la Licencia Creative Commons Atribuci�n 4.0
% Internacional. Para ver una copia de esta licencia, visita
% http://creativecommons.org/licenses/by/4.0/.

% Cultura videojuegos -> indsutria genera mucho dinero -> pirater�a

% Mecanimos genericos de anticopia, pero tambien evitar traducciones, drm y trampas en multijugador

% Ofuscacion e ingenier�a inversa

% Depuracion y desarrollo de herramientas que han servidor para ingeniera inversa,

% Resultados que ratifican una falta en algunos campos, de seguridad

%% Spanish abstract %%
\cleardoublepage
\thispagestyle{empty}
\selectlanguage{spanish}

\begin{center}
{\large\bfseries Mecanismos de seguridad en videojuegos} \\
Benito Palacios S�nchez \\
\end{center}

\vspace{0.7cm}
\noindent \textbf{Palabras clave:} seguridad, videojuegos, ingener�a inversa, DRM, Nintendo DS.

\vspace{0.7cm}
{\centering{}\textbf{Resumen}}

Este documento expone el an�lisis de seguridad realizado a diferentes juegos, principalmente de la videoconsola Nintendo DS.
Los videojuegos son parte de la cultura actual, utiliz�ndose en �mbitos educativos y de entretenimiento.
La industria detr�s de ellos est� en continuo crecimiento, creando m�s puestos de trabajo y generando mayores ganancias.


%% English abstract %%
\cleardoublepage
\thispagestyle{empty}
\selectlanguage{english}

\begin{center}
{\large\bfseries Security Mechanisms in Video Games} \\
Benito Palacios S�nchez \\
\end{center}

\vspace{0.7cm}
\noindent \textbf{Keywords:} security, video games, reverse engineering, DRM, Nintendo DS.

\vspace{0.7cm}
{\centering{}\textbf{Abstract}}

This is the english version of the abstract
