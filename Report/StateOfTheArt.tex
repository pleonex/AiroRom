% Copyright (c) 2015 Benito Palacios S�nchez - All Rights Reserved.
% Esta obra est� licenciada bajo la Licencia Creative Commons Atribuci�n 4.0
% Internacional. Para ver una copia de esta licencia, visita
% http://creativecommons.org/licenses/by/4.0/.

\chapter{Seguridad en videojuegos}
\label{sec:art}
Este trabajo explora los conceptos de seguridad y videojuegos.
Unos t�rminos que en principio no se suelen relacionar a no ser que se hable sobre la pirater�a.
Este fen�meno va en aumento a d�a de hoy, esperando un crecimiento del 22\% para 2015~\cite{Arxan}.
En esta misma fuente no solo trata la distribuci�n il�cita de contenidos y programas para romper los mecanismos anti-copia, tambi�n habla sobre la ingen�era inversa y de como `\textit{usando herramientas gen�ricas, los hackers pueden convertir r�pidamente binarios desprotegidos en c�digo fuente, volver a empaquetarlos y distribuirlos}'.

Com�nmente se asocia el t�rmino de \textit{hacker} a una persona que maliciosamente investiga un programa.
Es una mala interpretaci�n dada por medios y pel�culas, realmente se habr�a de hablar de \textit{cracker}~\cite{Xbox}.
El nombre de \textit{hacker} naci� en 1961, en los laboratorios del \ac{MIT}, para denominar a los estudiantes que dominaban con destreza la programaci�n.
A d�a de hoy, seg�n el RFC 1392\footnote{\url{https://tools.ietf.org/html/rfc1392}} se define \textit{hacker} como:

\foreignquote{english}{Hacker: A person who delights in having an intimate understanding of the internal workings of a system, computers and computer networks in particular. The term is often misused in a pejorative context, where `cracker' would be the correct term.}

\foreignquote{spanish}{Hacker: Persona apasionada en entender c�mo funciona, en detalle, internamente un conjunto de sistemas, ordenadores y redes de ordenadores. Generalmente se usa de forma incorrecta en un contexto peyorativo, en este caso el t�rmino correcto ser�a `cracker'.}

De esta forma, este mismo RFC define \textit{cracker} como:

\foreignquote{english}{Cracker: A cracker is an individual who attempts to access computer systems without authorization. These individuals are often malicious, as opposed to hackers, and have many means at their disposal for breaking into a system.}

\foreignquote{spanish}{Cracker: Individuo que intenta acceder a un sistema de ordenadores sin autorizaci�n. Estos individuos son generalmente maliciosos, en oposic�n a los `hackers', y tienen intereses ocultos en su intento por romper el sistema.}

Este trabajo muestra la seguridad de los juegos con el �nico prop�sito educacional, como menciona Andew Huang~\cite{Xbox}: \textit{For every copyright protection scheme that is defeated by a hacker, there is someone who learned an important lesson about how to make a better protection scheme.} (\textit{Para cada esquema de protecci�n copyright que un hacker rompe, hay alguien que aprende una importante lecci�n sobre c�mo hacer un esquema de protecci�n m�s robusto.}).

\section{Seguridad en videoconsolas}

\section{Mecanismos de seguridad}

\section{Ingenier�a inversa}
% Historia y objetivos de REDO, a Esprit project

% Definici�n de ingenier�a inversa por REDO

% Objetivos de la ingenier�a inversa por REDO

\subsection{Legalidad}
