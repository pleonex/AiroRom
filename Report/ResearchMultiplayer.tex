% Copyright (c) 2015 Benito Palacios S�nchez - All Rights Reserved.
% Esta obra est� licenciada bajo la Licencia Creative Commons Atribuci�n 4.0
% Internacional. Para ver una copia de esta licencia, visita
% http://creativecommons.org/licenses/by/4.0/.

\chapter{Multijugador y contenido descargable}


\section{Duet}
Por �ltimo se ver� el caso del juego \textit{Duet} para \textit{iOS} y sus niveles extras.
Se trata de una compra integrada en la aplicaci�n por 0.99 euros que a�ade un nuevo conjunto de pruebas.
Analizando de nuevo el contenido de la carpeta centr�ndose en estos niveles, se averigua que estos est�n presentes en el juego y que mediante la compra se activan.
La b�squeda se centra por tanto en saber d�nde se guarda la configuraci�n del juego para ver si est� protegida que como se ver� no es as�.

\includefigure{fig:mp-duet}%
{Filas de la base de datos de \textit{Duet} que activan el contenido extra.}%
{imgs/MP-Duet.png}

En la carpeta \textit{Documents} de la aplicaci�n existen tres bases de datos \textit{sqlite}.
Abriendo la de mayor tama�o, \textit{persistent-data.db}, y gracias a su nombre que nos indica que hay datos persistentes, es decir, constantes en cada ejecuci�n del juego nos encontramos las filas de la figura~\ref{fig:mp-duet}.
En ella se ve como los valores que activan el contenido extra, el contenido de pago, est�n puestos a \textsf{0}, el valor que corresponde a \textit{desactivado} por lo general.
Si se cambia a \textsf{1} y se inicia la aplicaci�n de nuevo nos encontraremos con este contenido activado.

La protecci�n ante este tipo de casos es tan sencilla como poner una contrase�a a la base de datos.
Se trata de un mecanismo que est� implementado nativamente\footnote{\url{http://web.archive.org/web/20070813071554/http://sqlite.phxsoftware.com/forums/t/130.aspx}} en las bibliotecas de \textit{sqlite} y muy sencillo de usar.
La contrase�a se puede almacenar en texto plano en la aplicaci�n, pues dificultar� la tarea de conocerla y al valer el contenido tan poco (menos de 1 euro) no compensar� el esfuerzo dedicado como se ha discutido en cap�tulos anteriores.

\section{Ejecutables firmados}
%http://problemkaputt.de/gbatek.htm#biosramusage
Si 0x27FFC40 est� a 1 indica que se carga de la ROM, se supone segura y no se comprueba el HMAC de los archivos.
En download play hay otro valor y se comprueba, como se hace en ARM9 se puede omitir.
Parece que se introdujo en DSi.
Evitar codigo no seguro cambiado en un man-in-the-middle para hackear la DSi.
El arm9 se firma durante el envio pero los datos no.
%http://problemkaputt.de/gbatek.htm#dswifimultiboot
Una vez que se ha ejecutado un juego desde una flashcard, el arm9 se puede editar pues el RSA se genera al vuelo sobre lo que hay y habria que cambiar el c�digo de los overlays guardado en el arm9 o deshabilitar el flag de la tabla de overlays.
