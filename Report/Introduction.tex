% Copyright (c) 2015 Benito Palacios S�nchez - All Rights Reserved.
% Esta obra est� licenciada bajo la Licencia Creative Commons Atribuci�n 4.0
% Internacional. Para ver una copia de esta licencia, visita
% http://creativecommons.org/licenses/by/4.0/.

\chapter{Introducci�n}
\section{Motivaci�n}
% Historia de los videojuegos -> d�a de hoy y millones -> seguridad en mecanismos en anticopia
El primer videojuego se remonta al a�o 1952~\cite{VG-History}, cuando el profesor de inform�tica Alexander S. Douglas creo un juego con gr�ficos para ordenador.
Llamado \textit{Nought and crosses} u \textit{OXO}, se trata de un tres en rayas implementado para la computadora \textit{EDSAC} de la Universidad de Cambridge.
Este permit�a enfrentar a una persona contra la m�quina.
Seis a�os m�s tarde, William Higginbotham desarroll� el juego \textit{Tennis for Two} sobre un oscilosopicio, inventando el primer videojuego multijugador.
Cuatro a�os despu�s, un estudiante del MIT (\textit{Massachusetts Institute of Technology}) creo el juego un juego con gr�ficos vectoriales donde dos naves se enfrentabas, denominado \textit{Spacewar}.

Desde entonces y 63 a�os m�s tarde, existen ocho generaciones de videoconsolas.
Las �ltimas permiten jugar en alta calidad y 60 fps, consiguendo un alto nivel de detalle y realismo.
La industria de los videojuegos ha evolucionado a un ritmo imparable siendo en Estados Unidos uno de los sectores que m�s r�pido ha crecido~\cite{ESA}.
En 2014, se vendieron solo en Estados Unidos 135 millones de juegos, generando unas ganancias de 22 mil millones de d�lares.
Para 2015, la empresa \textit{Gartner} estima que la venta de videojuegos a nivel global ser� de en torno a 100 mil millones de euros.
En cuanto a Espa�a, en 2014 la industria creci� un 21\% facturando 413 millones de euros seg�n la \textit{Asociaci�n Espa�ola de Empresas Productoras y Desarrolladoras de Videojuegos y Software de Entretenimiento}~\cite{DEV}.

% Enlazar con romhacking y foros

\section{Objetivos}
El objetivo que persigue este trabajo es el de ofrecer tres diferentes aspectos de seguridad sobre videojuegos no tan conocidos, del que no existe literatura al respecto.
En cada uno se ofrecer� varios an�lisis de juegos, mostrando sus vulnerabilidades y puntos fuertes y ofreciendo recomendaciones.

Se ha escogido juegos de la \acl{NDS} para la mayor�a de los casos, por la documentaci�n sobre el hardware de la consola que existe y, emuladores avanzados.
Tambi�n se han analizado dos juegos de plataformas m�viles.
El objetivo ser� el de analizar los ficheros buscando, seg�n la problem�tica, un algoritmo que proteja al recurso o su ausencia.
Para ello se depurar� el juego y desarrollar� software para de ayuda.

El primer aspecto ser�n las traducciones no oficiales realizadas por personas ajenas a una empresa.
Realizadas de forma altruista por usuarios,

\section{Organizaci�n}
