% Copyright (c) 2015 Benito Palacios S�nchez - All Rights Reserved.
% Esta obra est� licenciada bajo la Licencia Creative Commons Atribuci�n 4.0
% Internacional. Para ver una copia de esta licencia, visita
% http://creativecommons.org/licenses/by/4.0/.

\chapter{Conclusiones}
\label{sec:conclusion}
% De qu� va el cap�tulo
A lo largo del trabajo se han analizado una serie de juegos, estudiando los mecanismos de protecci�n que implementan y se�alando sus carencias y debilidades.
Todos los objetivos propuestos han sido alcanzados, estudiando un total de 21 juegos.
De estos, se han mencionado en esta memoria un total de 14 y documentado el resto en la \textit{wiki} del repositorio del trabajo\footnote{\url{https://github.com/pleonex/airorom/wiki/Mecanismos-a-investigar}}.

Se plante� un objetivo opcional que, por falta de tiempo, no se ha podido cumplir.
Este fue crear un depurador de c�digo para \acl{NDS} llamado \textit{NitroDebugger}\footnote{\url{https://github.com/pleonex/NitroDebugger}}, del cual se ha terminado su n�cleo y faltar�a crear una interfaz gr�fica y un programa desensamblador.
Este proyecto se present� al \textit{Certamen de Proyectos Libres de la UGR 2014}. A pesar de no estar entre los finalistas debido al estado de completitud, recibi� buenas cr�ticas del jurado\footnote{\url{https://goo.gl/g8xdWZ}}.

En cuanto a objetivos acad�micos, se han alcanzado los siguientes:
\begin{itemize}
    \item Identificar problemas no tratados en la literatura.

    \item Organizar un trabajo en tareas e investigarlas.

    \item Desarrollar software adicional en los lenguajes \texttt{C\#} y \texttt{python} para complementar la investigaci�n.

    \item Utilizar programa de control de versiones (\texttt{git}).

    \item Aprender y usar conceptos de bajo nivel de software y hardware, estudiando protocolos y componentes de la \acl{NDS}, as� como su lenguaje ensamblador, \texttt{ARM}.

    \item Dise�ar metodolog�as de ingenier�a inversa para analizar videojuegos.

     \item Dise�ar estrategias para estudiar las comunicaciones inal�mbricas, como han sido modificar el emulador DeSmuME para guardar los paquetes y realizar un dise�o \textit{man-in-the-middle} para capturar el tr�fico de las aplicaciones m�viles.

     \item Aprender el lenguaje \LaTeX para la redacci�n de esta memoria.
\end{itemize}

\section{Trabajo futuro}
Los estudios de este trabajo han pretendido resumir los mecanismos m�s frecuentes encontrados en los juegos de \ac{NDS} as� como se�alar las carencias en plataformas m�viles.
A continuaci�n se ofrece una lista sobre t�picos en los que los que se podr�a profundizar:

\begin{itemize}
    \item Estudiar los mecanismos de seguridad implementados en las videoconsolas.
    \item Estudiar los mecanismos anti-copia implementados f�sica y digitalmente sobre los videojuegos y aplicaciones m�viles.
    \item Desarrollar un explorador de juegos avanzado, pudiendo detectar algoritmos y formatos ya estudiados.
    \item Terminar la implementaci�n del depurador de c�digo remoto.
    \item Analizar videojuegos de la nueva generaci�n de consolas: \acl{N3DS}, Wii U, \acl{PS4} y \acl{XOne}.
    \item Realizar este estudio sobre aplicaciones de ordenador.
    \item Estudiar los algoritmos \ac{DRM} que aplican plataformas como \textit{Steam}.
    \item Implementar algoritmos propuestos en videojuegos y dispositivos m�viles.
    \item Estudiar protocolos usados por videojuegos de aplicaciones m�viles para realizar micropagos.
    \item Estudiar los \textit{exploits} que las \textit{flashcard} para \acl{NDS}, \acl{DSi} y \acl{N3DS} utilizan.
    \item Estudiar la integridad en los archivos de guardado (relacionado con el punto anterior).
\end{itemize}
