% Copyright (c) 2015 Benito Palacios S�nchez - All Rights Reserved.
% Esta obra est� licenciada bajo la Licencia Creative Commons Atribuci�n 4.0
% Internacional. Para ver una copia de esta licencia, visita
% http://creativecommons.org/licenses/by/4.0/.

\chapter{Seguridad y derechos de autor}

A d�a de hoy uno de las principales preocupaciones de la industria respecto a la difusi�n de su obra se centra en los derechos de autor.
Esto es especialmente notable en el mundo de los videojuegos acosado por una pirater�a continua que causa p�rdidas millonarias de dinero\footnote{http://drm.web.unc.edu/games}.
De este contexto nace la idea del \ac{DRM}.
Se trata en un mecanismo con el �nico fin de imposibilitar la distribuci�n no autorizada de contenido.
Existe mecanismos desde los m�s sencillos y antiguos como son las claves que vienen impresas en un CD a los m�s nuevos que incluyen el cifrado del videojuego con datos personales de la cuenta del usuario\footnote{http://www.darthnull.org/2014/10/06/ios-encryption} o una conexi�n necesaria a los servidores de las compa��a.
Las caracter�sticas comunes de estas t�cnicas se pueden resumir en los siguientes puntos\footnote{http://es.wikipedia.org/wiki/Gesti�n_digital_de_derechos}.

\begin{itemize}
    \item Detectar qui�n accede a cada obra, cu�ndo y bajo qu� condiciones.
    \item Autorizar o denegar de manera inapelable el acceso a la obra.
    \item Autorizan el acceso bajo condiciones restrictivas.
\end{itemize}

Estos mecanismos a nivel general se basan en \textit{software} siendo lo frecuente el cifrado de contenido de forma que solo las \emph{aplicaciones autorizadas} tienen las claves y algoritmos necesarios para acceder a �l.
En otros casos, es el propio sistema operativo quien se encarga de esta tarea, esto se encuentra generalmente en entornos cerrados donde no es posible ejecutar aplicaciones no autorizadas para acceder a la memoria RAM donde se encontrar�a la aplicaci�n.\footnote{Este es el caso del sistema operativo iOS donde el cifrado de la aplicaci�n la realiza el \textit{kernel}. Una vez se pudo ejecutar aplicaciones no firmadas, fue f�cil descifrar el contenido de terceras aplicaciones leyendo la memoria RAM.}.

El principal problema es la posibilidad de aplicar m�todos de \emph{ingenier�a inversa} sobre el \textit{software} autorizado. Una vez conocido c�mo una aplicaci�n accede al contenido protegido, nada imposibilitar crear una versi�n que tambi�n lo permita.
